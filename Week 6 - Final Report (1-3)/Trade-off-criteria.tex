\section{Trade-off Criteria}
The main criteria of the trade-off are introduced and detailed in this section. In general, if a trade-off criterion is indifferent for all concepts, it is eliminated as a trade-off criterion. The main criteria are Cost, Public Space, Performance, Sustainability, Noise, Safety, User interaction, Stability \& Control, and Maintenance. These criteria were split up into sub-criteria in the final trade-off table.

\subsection{Cost}
For cost, one of the important costs under consideration is the manufacturing cost. The main costs in manufacturing are the material and component costs, which are both considered in the trade-off. For the material cost, a qualitative analysis is done, as exact numbers are hard to determine in this early design stage. This analysis is based on cost of the material itself, the tooling and labour intensity, and also the proportions of waste during the processing of those materials. The energy cost is considered by comparing the value of the energy carriers. In the case of a battery, this is the electricity cost directly, and in hydrogen this is the energetic cost of hydrogen.

\subsection{Public space}
The space the system will take in a public space is considered in the trade-off, as a too large occupancy is not feasible in urban areas. Especially the deployed state of the system is a matter of interest. This area has been calculated by finding the maximum x- and y-value combinations and projecting it to the ground. Circular areas have been obtained for which the outermost radius was that of the rotors in the helipack and the ice cream cone (which were equal and lowest). For the quadcopter, the area has been treated as a rectangle with rounded corners. 


\subsection{Performance}
% Stuff about drag area
Aerodynamic drag is an inherent factor in the performance of any flying vehicle. For all concepts, the drag force is determined at the maximum cruise speed. This is, however, a parameter influenced by too many conditions. Therefore, the drag area ($S_{ref}C_D$) is used for the comparison. This factor is calculated by the drag tool described in \autoref{ch:aerodynamica}. These drag areas are inverted, as the lowest values is the best.

Performance is also assessed in the propulsive efficiency. Since the entire system is tentative, the total energy used by the propulsive system over the entire flight is compared to the stored energy to also take the powertrain and storage efficiency into consideration. For both types of configuration, this takes into consideration the loss of energy due to it needing to power the support systems.
The propeller efficiency is also checked: based on literature and their sizes. Larger propellers have larger efficiencies. On top of that, the coaxial fan has an extra added propeller efficiency, and the fan a slight increase as well \cite{rotaryWingAero}.
In the case of the battery-powered system, the total propulsive efficiency takes into consideration the electric conversion and controlling losses. For the fuel cell, all auxiliary system draw is projected.


\subsection{Sustainability}
For sustainability three major factors are considered for the trade-off: indirect emissions, end of life purposes and modularity. These factors are then quantified for each concept. To quantify indirect emissions, Life cycle assessment is preformed for each concept. The steps taken to perform this assessment is elaborated in detail in \autoref{subsec:sustaintradeoff}. 

To quantify end of life, recyclability and re-usability of each concept is considered. At this stage of the design process, focus is only laid on the power system and materials used. Percentage of material that can be reused and recycle from power system and structure for each concept is calculated using literature. This is then used to calculate a recycling value for each concept which is used for the trade-off. A detailed description of this process is given in \autoref{subsec:EOLstrategies}. Moreover, the modularity of each concept is assessed. A scale of 1-5 has been used to quantify each concept relative to the others. A score of 1 means that components of a concept cannot be used for multiple purposes at all, whereas a score of 5 means multiple purposes are assured for a single component. 3 aspects have been assessed, namely the types of failure, which defines the possibilities for other functions. Secondly, re-purposes of individual parts are described for other functions. Also the part extraction, assessing mechanical fasteners and adhesives, has been a score from 1-5 for each concept. 

\subsection{Noise}
Assessing the noise generated by the system in its current state proved to be very hard. Empirical relationships could be found for propeller-driven aircraft, which took known parameters as inputs- the outputs of these equations when tried with values for existing helicopters however were over 40\% too high. Other such equations were found for helicopters, but they only held for larger aircraft: for smaller configurations, this had a 50\% inaccuracy. Hence, another way had to be found to assess the noise generated by the system.
In this comparison, only the noise generated by the main rotor(s) will be taken into consideration- all other noise emission is considered negligible to this. For a rotorcraft, the sound generated is largely related to the rotor tip speed \cite{quietheli}. Ongoing research attempts to lower this tip speed while retaining thrust generated to lower sound. This is why the rotor tip speed is chosen as a scale for noise generated. It must be noted that this does not cover for the quadcopter configuration, which uses a ducted fan. In a a study conducted on a rotor similar to ours in size, using a ducted fan reduced noise generated by about 35\% \cite{ductedFanNoise}. To Account for this, the value used for concept one is reduced by a slightly more conservative 25\%.
Note that these values are largely dependent on the propeller RPM: The RPM in question being chosen to generate enough lift with the configuration as described in \autoref{sec:3pSizingProcess}.

\subsection{Safety}
%\subsubsection{\textbf{User safety}}
In the assessment of user safety of each concept, crash-worthiness was considered to be biggest factor. A United States army report on the crash-worthiness of helicopters was used  and translated to our concepts. The two most relevant aspects of crash-worthiness were container strength and post crash fire \webcite{USAARL}. Container strength being connected to how well the user is in a protective cell. These two criteria were evaluated on a qualitative basis, since the stage of the design does not allow for a full crash strength analysis calculation. The container strength, post crash fire, rotor shielding and encapsulation criteria were weighted equally and both graded on a scale of 1 to 5 based on a group discussion of the concepts. The ice cream cone was given a 1 for post crash fire due to the hydrogen tank as fuel storage and the other two concepts were given a 4/5, since the batteries used are not sensitive for catching fire and if they do, they are not explosive. For container strength the ice cream cone was given a 4/5 since the user is in a closed structure, it is not a 5 however since the cone and bubble are not designed for big impacts. The quadcopter was given a 3 since the beams with rotors form a buckle zone around the user for direct impact. Also for vertical impact, the user is on top of the vehicle and thus the vehicle can take up most of the impact. The helicopter was ranked a 1 since the user is not at all protected. The rotor shielding criteria was scored highest on by concept 3, since the user is inside the body and can therefore not get in contact with the rotors. In concept 2 this is more the case, as the rotors are turning directly above the head of the user, hence a score of 2/5. The quadcopter has ducted fans around the rotors and also the rotors are located out of arms reach of the user, 3/5. Lastly the encapsulation criteria was treated. Concept 3 provides full encapsulation for the user, giving it 5/5. The helipack scores a 1 in this aspect, as again there is not encapsulation to protect against wind or rain. The quadcopter will have a windscreen like a motorcycle, therefore providing some protection from the environment giving it a 3/5. 
All these safety criteria scores were added and normalized. The final scoring for the trade-off is: Concept 1: 90.77, concept 2: 56.92, concept 3: 100. 


\subsection{User interaction}
For user interaction two factors are taken into consideration for the trade-off: user comfort and user visibility. For user comfort, ergonomics and weather protection of each concept in analyzed. As it is not possible to quantify these characteristics, a qualitative analysis is performed for each concept. From the analysis a relative score is given to each concept according to the performance which are used in the trade-off. The detailed description of this analysis is given in \autoref{sec:comtrafeoff}. For visibility two viewing angles are taken into consideration: side view and top view. Top view is defined as the maximum horizontal viewing angle of the user and side view is the maximum vertical viewing angle of the user. Both of these angles are calculated for the for each concept and given in \autoref{sec:safety tradeoff}. Which are then used in the trade-off.

\subsection{Stability \& Control}

When it comes to the stability \& control subsystem, an important trade-off criteria are defined: the controllability versus automation level. The approach of quantifying these concepts is detailed in this section. 

An important requirement for the system is that the system shall be controllable by a smart device. This means that controlling the system should be relatively simple, as a highly coupled control system requires a skilled pilot to control (more about flight training can be found in \autoref{sec:training}). An inverse relationship between control complexity and automation level is observed. An overall complexity of the control system will be given to all concepts. This will be done by adding all the coupled control influences.

For the Quadcopter, differential thrust of 4 rotors is used. When pitching, the pitching acceleration, the horizontal and vertical acceleration is influenced: a complexity of 3. Furthermore, when rolling, the rolling acceleration, the vertical acceleration and the acceleration in y-direction is influenced (another 3). Yawing can be done independently of roll and pitch, but it influences the vertical acceleration (another 2). This results in a control complexity of 8.

For the Helipack, cyclic control, collective control and thrust control is done using twin coaxial rotors. Collective control only influences the vertical acceleration. When performing cyclic control, the pitch acceleration, the longitudinal and vertical acceleration is influenced. In rolling condition, the rolling acceleration the vertical acceleration and the acceleration in y-direction are influenced. As for the Quadrotor, yawing can be done independently from rolling and pitching but it influences the vertical acceleration. Hence, this concept has a control complexity of 9.

For the Ice Cream Cone, the same control methods as for the Helipack are used, except that instead of a coaxial rotor, this concept features a tail rotor. When the system is in pitch, besides pitching acceleration, longitudinal and vertical acceleration, there will also be a rolling acceleration due to gyroscopic precession. For rolling, there will be a net net rolling acceleration, a lateral and vertical acceleration and a yawing acceleration. Using the collective control will produce a vertical acceleration and a net yawing acceleration (due to the driving torque of the main rotor). The tail collective control produces a yawing acceleration, a lateral acceleration and a rolling acceleration (that is if there is a vertical distance between the COG and the tail rotor). This results in a control system complexity of 12. 

Note that for a one-to-one control system (6 control methods for 6 degrees of freedom) would result in a perfect system with the lowest complexity. Therefore, in order to make the stability and control part of the trade-off sensitive to changes, every value has been subtracted by 6.

\subsection{Maintenance}
For the maintenance trade criterion the expected maintenance sensitivity of the concepts was based on the complexity of each design. Moving parts were considered for this. This included moving parts during operation, but also components of deployment mechanisms of the concepts, like the retractable bubble and cone in concept 3. The higher the number of moving parts, the more complex the concept and thus a bad maintenance score. The quadcopter scored 16 moving parts and joints, due to the folding mechanism of the rotor arms. The helipack only 1 part, the propeller. Concept 3 has 7 moving parts, including the retractable bubble and cone.